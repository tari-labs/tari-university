

 

## Appendix C: Proof of Theorem 1 



**Notation**

The letters  $j$,  $k$  and  $ l$  denote non-negative integers unless otherwise stated. In addition,  $n = 2^k$, $ i \in \{ 0, 1, 2, ... , n-1 \}$, and  $j \in \{ 1, 2, ... , k \}$. 

The multiplicative identity of the field  $\mathbb{F}_p$  is denoted by  $1_{\mathbb{F}_p}$. 

The IPP verifier's  $j-$th challenge is denoted by  $u_j$  and its parity exponent is defined by,   $b(i,j) = \begin{cases} {-1} & {\text{if}\ \ (i\ \ mod\ \ 2^j) < 2^{j-1}} \\ {+1} & {\text{if}\ \ (i\ \ mod\ \ 2^j) \geq 2^{j-1}} \end{cases} $  



The proof of **Theorem 1** makes use of a few basic properties of the parity values  $b(i,j)$  captured here in the form of Lemmas 1 and  3, and their corollaries. 

These Lemmas and their corollaries follow readily from the definition of the parity exponent  $b(i,j)$  of verifier challenges and their multiplicative inverses. 

The next lemma is a well-known fact entailed in undergraduate Mathematics text books, mostly in tables of formulas.  

**Lemma 1**

For an indeterminate  $x$ , we have  $ \ \ x^k - 1 = (x - 1)(x^{k-1} + x^{k-2} + ... + x + 1) .$ 



**Corollary 1**

(a) $ \ \ 2^k - 1\ \ =\ \ 2^{k-1} + 2^{k-2} + \dots + 2 + 1$. 

(b) $ \ \ 2^k - 1\ \ \geq\ \ 2^{k-1}\ \ $ for all $ k \geq 1$. 

(c) $ \ \ (2^l - 1) \text{ mod } 2^j = 2^{j-1} + 2^{j-2} + \dots +  2 + 1\ \ $  for any  $l \geq j$.

(d) $ \ \ ((n - 1) - 2^{j-1}) \text{ mod } 2^j = 2^{j-2} + 2^{j-3} + \dots +  2 + 1 $.

(e) $ \ \ i\ \  =\ \ c_{l-1} \cdot 2^{l-1} + c_{l-2} \cdot 2^{l-2} + \dots + c_1 \cdot 2 + c_0\ \ $ for  $ l < k $  and some  $ c_{l-i} \in \{ 0 , 1 \}$.



**Lemma 2**

(a) $ \ \ b(0,j) = -1 $. 

(b) $ \ \ b(1,1) = +1 $.

(c) $ \ \ b(n-2,1) = -1 $.

(d) $ \ \ b(1,j) = -1 ,\ \ \forall\ \ j > 1 $.

(e) $ \ \ b(n-1,j) = +1 $.

(f) $ \ \ b(n-2,j) = +1 ,\ \ \forall\ \ j > 1 $.

(g) $ \ \ b( 2^j , j ) = -1  $.

(h) $ \ \ b( 2^{j-1} , j ) = +1 $. 

(i) $ \ \ b(2^l,j) = -1,\ \ \forall\ \ l < j-1  $.

(j) $  \ \ b(2^l,j) = -1,\ \ \forall\ \ l > j $.

(k) $ \ \ b((n-1)-2^{j-1}, j) = -1 $.



**Corollary 2** 

(a) $ \ \ b(0,j) = (-1) \cdot b(n-1,j)\ \ \text{ for all } j $.  

(b) $ \ \ b(i,j)  = (-1) \cdot b( (n-1)-i , j ) \ \ \text{ for all }  i \text{ and for all  }  j $.  

(c) $ \ \ b( 2^{j-1} , j ) = b(n-1,j)\ \ \text{ for all } j $ . 

(d) $ \ \ b(0,j) = b((n-1)-2^{j-1}, j)\ \ \text{ for all } j $. 

***Proof of Corollary 2, Part (b)***

By induction on  $k$ , where  $j \in \{ 1, 2, 3, \dots , k \} $. 

For  $j = 1$  where  $ i $  is *even*:  Note that  $ i \text{ mod } 2^1  = 0 < 2^0 = 1 $ ,  and thus  $ b(i,1) = -1 $. On the other hand  $ ((n-1)-i)$  is *odd*, hence  $ ((n-1)-i) \text{ mod } 2^1 = 1 = 2^0 $. So that  $ b((n-1)-i, j) = +1 $ . 

For  $j = 1$  where  $ i $  is *odd*:  Similarly  $ i \text{ mod } 2^1  = 1 = 2^0 $ and  $ b(i,1) = +1 $ .  Since  $  ((n-1)-i) $  is *even*,  $ ((n-1)-i) \text{ mod } 2^1 = 0 < 2^0 $. And therefore  $ b((n-1)-i, j) = -1 $. 

This proves the base case. i.e.,  $ b(i,1) = (-1) \cdot b((n-1)-i, j) $. 

Now for  $j > 1$. Note that by Part (e) of Corollary 1,  $\ \ i  \text{ mod } 2^j  = c_{j-1} \cdot 2^{j-1} + c_{j-2} \cdot 2^{j-2} + \dots + c_1 \cdot 2 + c_0  $  and   $ ((n-1)-i) \text{ mod } 2^j = (2^{j-1} + 2^{j-2} + \dots +  2 + 1) - (c_{j-1} \cdot 2^{j-1} + c_{j-2} \cdot 2^{j-2} + \dots + c_1 \cdot 2 + c_0) $. 

Suppose that  $ b(i, j) = +1 $. Then  $ c_{j-1} \cdot 2^{j-1} + c_{j-2} \cdot 2^{j-2} + \dots + c_1 \cdot 2 + c_0 \geq 2^{j-1}$, which means  $ c_{j-1}  =  1$. This implies  $ ((n-1)-i) \text{ mod } 2^j = (2^{j-2} + 2^{j-3} + \dots +  2 + 1) - (c_{j-2} \cdot 2^{j-2} + c_{j-3} \cdot 2^{j-3} + \dots + c_1 \cdot 2 + c_0) < 2^{j-1} $. Yielding  $ b((n-1)-i) = -1 $. The converse argument is the same. 

Suppose that  $  b(i, j) = -1 $. Then  $ c_{j-1} \cdot 2^{j-1} + c_{j-2} \cdot 2^{j-2} + \dots + c_1 \cdot 2 + c_0 < 2^{j-1} $ which means  $ c_{j-1}  = 0 $. This also implies $  ((n-1)-i) \text{ mod } 2^j \geq 2^{j-1} $.  Hence  $ b((n-1)-i) = +1 $.  Again, the converse argument here follows the reverse argument. 

The above two cases prove the inductive step. i.e., $  b(i, j) = (-1) \cdot b((n-1)-i) $.  $ \ \ \Box $





**Theorem 1 [Some properties of the set of coefficients $\{ s_i \}$]** 

Let  $s_i = \prod\limits_{j = 1}^k u_j^{b(i,j)}$  be the coefficient of  $G_i$  the $i-$th component of the initial IPP input vector $\mathbf{G} = ( G_0 , G_1 , G_2 , ... , G_{n-1})$.  Then, 

1.	$\ \ s_i \cdot s_{(n-1) - i} = 1_{\mathbb{F}_p}\ \ $ for all $i \in \{ 0, 1, 2, ... , n-1 \}$.

2.	$\ \ s_{2^{(j-1)}} \cdot s_{n-1} = u_j^2\ \ $ for all $j \in \{ 1, 2, 3, ... , k \}$.

3.	$\ \ s_0 \cdot s_{(n-1) - 2^{(j-1)}} = u_j^{-2}\ \ $ for all $j \in \{ 1, 2, 3, ... , k \}$.



**Proof of Theorem 1** 

1. By induction on  $n$ , where  $ i \in \{ 0, 1, 2, \dots , n-1 \} $.  

   For  $ i = 0 $. By Part (a)  of  Corollary 2,  $\ \ b(0,j) = (-1) \cdot b(n-1,j)\ \ \text{ for all } j $ . But this holds true *if and only if*  $ \  \  u_j^{b(0,j)} = \Big( u_j^{b(n-1,j)} \Big)^{-1} $.  And hence  $ s_0 \cdot s_{n-1}  =  1_{\mathbb{F}_p} $,  proving the base case. 

   The inductive step: Suppose  $ s\_{i-1} \cdot s\_{(n-1) - (i-1)}  =  1\_{\mathbb{F}_p}  .\ \ $  And now,  $$ s_i \cdot s\_{(n-1)-i} = \big( s\_{i-1} \cdot s\_{(n-1) - (i-1)} \big) \cdot u_j^{b(0,j)} \cdot u_j^{b(n-1,j)} .$$  

   By the inductive step, this yields,  $$ s_i \cdot s_{(n-1)-i} = 1_{\mathbb{F}_p} \cdot u_j^{b(0,j)} \cdot u_j^{b(n-1,j)} .$$ 

   According to Part (b) of Corollary 2,  $b(i,j) = (-1) \cdot b((n-1)-i,j)$. Which holds true *if and only if*  $\ \  u_j^{b(0,j)} = \Big( u_j^{b(n-1,j)} \Big)^{-1} .$  It therefore follows that  $  s_i \cdot s\_{(n-1)-i} = 1\_{\mathbb{F}_p} \cdot u_j^{b(0,j)} \cdot u_j^{b(n-1,j)} = 1\_{\mathbb{F}_p} \cdot 1\_{\mathbb{F}_p} = 1\_{\mathbb{F}_p}$. 

2. This part follows readily from  Part (c)  of  Corollary 2.      

3. This part also follows readily from  Part (d)  of  Corollary 2.    $ \ \ \Box$  


