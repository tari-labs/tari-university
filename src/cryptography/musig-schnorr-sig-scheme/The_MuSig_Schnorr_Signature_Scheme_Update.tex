%% LyX 2.3.1-1 created this file.  For more info, see http://www.lyx.org/.
%% Do not edit unless you really know what you are doing.
\documentclass[english]{article}
\usepackage[T1]{fontenc}
\usepackage[latin9]{inputenc}
\usepackage{amstext}
\usepackage{stackrel}
\usepackage{babel}
\begin{document}
\title{\textbf{The MuSig Schnorr Signature Scheme}}
\maketitle
\noindent \begin{flushleft}
This report investigates Schnorr Multi-Signature Schemes (MuSig),
which makes use of key aggregation and is provably secure in the \emph{plain
public-key model}. \linebreak{}
\par\end{flushleft}

\noindent \begin{flushleft}
Signature aggregation involves mathematically combining several signatures
into a single signature, without having to prove Knowledge of Secret
Keys (KOSK). This is known as the \emph{plain public-key model} where
the only requirement is that each potential signer has a public key.
The KOSK scheme requires that users prove knowledge (or possession)
of the secret key during public key registration with a certification
authority, and is one way to generically prevent rogue-key attacks.\linebreak{}
\par\end{flushleft}

\noindent \begin{flushleft}
Multi-signatures are a form of technology used to add multiple participants
to cryptocurrency transactions. A traditional multi-signature protocol
allows a group of signers to produce a joint multi-signature on a
common message. 
\par\end{flushleft}

\tableofcontents{}

\section{Introduction }

\subsection{Schnorr signatures and their attack vectors }
\noindent \begin{flushleft}
Schnorr signatures produce a smaller on-chain size, support faster
validation and have better privacy. They natively allow for combining
multiple signatures into one through aggregation and they permit more
complex spending policies. \linebreak{}
\par\end{flushleft}

\noindent \begin{flushleft}
Signature aggregation also has its challenges. This includes the rogue-key
attack, where a participant steals funds using a specifically constructed
key. Although this is easily solved for simple multi-signatures through
an enrollment procedure which involves the keys signing themselves,
supporting it across multiple inputs of a transaction requires \emph{plain
public-key security}, meaning there is no setup. \linebreak{}
\par\end{flushleft}

\noindent \begin{flushleft}
There is an additional attack, termed the Russel attack, after Russel
O'Connor, who has discovered that for multi-party schemes a party
could claim ownership of someone else's private key and so spend the
other outputs. \linebreak{}
\par\end{flushleft}

\noindent \begin{flushleft}
Wuille P., \cite{Wuille2018} has been able to address some of these
issues and has provided a solution which refines the Bellare-Neven
(BN) scheme. He also discussed the performance improvements that were
implemented for the scaler multiplication of the BN scheme and how
they enable batch validation on the blockchain. \cite{Blocks2018}\pagebreak{}
\par\end{flushleft}

\subsection{MuSig}
\noindent \begin{flushleft}
Introduced by Itakura \emph{et al.} \cite{Itakura1983}, multi-signature
protocols allow a group of signers (that individually possess their
own private/public key pair) to produce a single signature $\sigma$
on a message $m$. Verification of the given signature $\sigma$ can
be publicly performed given the message and the set of public keys
of all signers. \linebreak{}
\par\end{flushleft}

\noindent \begin{flushleft}
A simple way to change a standard signature scheme into a multi-signature
scheme is to have each signer produce a stand-alone signature for
$m$ with its private key and to then concatenate all individual signatures.
\linebreak{}
\par\end{flushleft}

\noindent \begin{flushleft}
The transformation of a standard signature scheme to a multi-signature
scheme needs to useful and practical, thus the newly calculated multi-signature
scheme must produce signatures where the size is independent of the
number of signers and similar to that of the original signature scheme.
\cite{Maxwell2018}\linebreak{}
\par\end{flushleft}

\noindent \begin{flushleft}
A traditional multi-signature scheme is a combination of a signing
and verification algorithm, where multiple signers (each with their
own private/public key) jointly sign a single message, resulting in
a combined signature. This can then be verified by anyone knowing
the message and the public keys of the signers, where a trusted setup
with KOSK is a requirement. \linebreak{}
\par\end{flushleft}

\noindent \begin{flushleft}
MuSig is a multi-signature scheme is novel in combining:
\par\end{flushleft}
\begin{enumerate}
\item \begin{flushleft}
Support for key aggregation; 
\par\end{flushleft}
\item \begin{flushleft}
Security in the \emph{plain public-key model}. \linebreak{}
\par\end{flushleft}

\end{enumerate}
\noindent \begin{flushleft}
There are two versions of MuSig, that are provably secure, which differ
based on the number of communication rounds:
\par\end{flushleft}
\begin{enumerate}
\item \begin{flushleft}
Three-round MuSig only relies on the Discrete Logarithm (DL) assumption,
on which ECDSA (Elliptic Curve Digital Signature Algorithm) also relies 
\par\end{flushleft}
\item \begin{flushleft}
Two-round MuSig instead relies on the slightly stronger One-More Discrete
Logarithm (OMDL) assumption
\par\end{flushleft}
\end{enumerate}

\subsection{Key aggregation}
\noindent \begin{flushleft}
The term \emph{key aggregation} refers to multi-signatures that look
like a single-key signature, but with respect to an aggregated public
key that is a function of only the participants' public keys. Thus,
verifiers do not require the knowledge of the original participants'
public keys- they can just be given the aggregated key. In some use
cases, this leads to better privacy and performance. Thus, MuSig is
effectively a key aggregation scheme for Schnorr signatures. \linebreak{}
\par\end{flushleft}

\noindent \begin{flushleft}
To make the traditional approach more effective and without needing
a trusted setup, a multi-signature scheme must provide sub-linear
signature aggregation along with the following properties: 
\par\end{flushleft}
\begin{itemize}
\item \begin{flushleft}
It must be provably secure in the \emph{plain public-key model }
\par\end{flushleft}
\item \begin{flushleft}
It must satisfy the normal Schnorr equation, whereby the resulting
signature can be written as a function of a combination of the public
keys
\par\end{flushleft}
\item \begin{flushleft}
It must allow for Interactive Aggregate Signatures (IAS) where the
signers are required to cooperate
\par\end{flushleft}
\item \begin{flushleft}
It must allow for Non-interactive Aggregate Signatures (NAS) where
the aggregation can be done by anyone
\par\end{flushleft}
\item \begin{flushleft}
It must allow each signer to sign the same message
\par\end{flushleft}
\item \begin{flushleft}
It must allow each signer to sign their own message
\par\end{flushleft}

\end{itemize}
\noindent \begin{flushleft}
This is different to a normal multi-signature scheme where one message
is signed by all. MuSig provides all of those properties.\linebreak{}
\par\end{flushleft}

\noindent \begin{flushleft}
There are other multi-signature schemes that already exist that provide
key aggregation for Schnorr signatures, however they come with some
limitations, such as needing to verify that participants actually
have the private key corresponding to the pubic keys that they claim
to have. \emph{Security in the plain public-key model }means that
no limitations exist. All that is needed from the participants is
their public keys. \cite{Wuille2018} 
\par\end{flushleft}

\section{Overview of multi-signatures }
\noindent \begin{flushleft}
Recently the most obvious use case for multi-signatures is with regards
to Bitcoin, where it can function as a more efficient replacement
of \emph{$n-of-n$} multisig scripts (where the signatures required
to spend and the signatures possible are equal in quantity) and other
policies that permit a number of possible combinations of keys. \linebreak{}
\par\end{flushleft}

\noindent \begin{flushleft}
A key aggregation scheme also lets one reduce the number of public
keys per input to one, as a user can send coins to the aggregate of
all involved keys rather than including them all in the script. This
leads to smaller on-chain footprint, faster validation, and better
privacy. \linebreak{}
\par\end{flushleft}

\noindent \begin{flushleft}
Instead of creating restrictions with one signature per input, one
signature can be used for the entire transaction. Traditionally key
aggregation cannot be used across multiple inputs, as the public keys
are committed to by the outputs, and those can be spent independently.
MuSig can be used here (with key aggregation done by the verifier).
\linebreak{}
\par\end{flushleft}

\noindent \begin{flushleft}
No non-interactive aggregation scheme is known that only relies on
the DL assumption, but interactive schemes are trivial to construct
where a multi-signature scheme has every participant sign the concatenation
of all messages. Maxwell G., \emph{et al. }\cite{Maxwell2018} focusing
on key aggregation for Schnorr Signatures and shows that this is not
always a desirable construction, and gives an IAS variant of BN with
better properties instead. \cite{Wuille2018}
\par\end{flushleft}

\subsection{Bitcoin $m-of-n$ multi-signatures }

\noindent Currently, standard transactions on the Bitcoin network
can be referred to as single-signature transactions, as they require
only one signature, from the owner of the private key associated with
the Bitcoin address. However, the Bitcoin network supports much more
complicated transactions which can require the signatures of multiple
people before the funds can be transferred. These are often referred
to as $m-of-n$ transactions, where m represents the amount of signatures
required to spend, while n represents the amount of signatures possible.
\cite{Contributors2017} 

\subsubsection{Use cases for $m-of-n$ multi-signatures}
\noindent \begin{flushleft}
When $m=1$ and $n>1$ it is considered a shared wallet, which could
be used for small group funds that do not require much security. It
is the least secure multi-sig option because it is not multi-factor.
Any compromised individual would jeopardize the entire group. Examples
of use cases include funds for a weekend or evening event, or a shared
wallet for some kind of game. Besides being convenient to spend from,
the only benefit of this setup is that all but one of the backup/password
pairs could be lost and all of the funds would be recoverable. \linebreak{}
\par\end{flushleft}

\noindent \begin{flushleft}
When $m=n$, it is considered a partner wallet, which brings with
it some nervousness as no keys can be lost. As the number of signatures
required increases, the risk also increases. This type of multi-signature
can be considered as a hard multi-factor authentication. \linebreak{}
\par\end{flushleft}

\noindent \begin{flushleft}
When $m<0.5n$, it is considered a buddy account, which could be used
for spending from corporate group funds. Consequence for the colluding
minority need to greater than possible benefits. It is considered
less convenient than a shared wallet, but much more secure. \linebreak{}
\par\end{flushleft}

\noindent \begin{flushleft}
When $m>0.5n$, a consensus account is termed. The classic multi-signature
wallet is a 2 of 3 and is a special case of a consensus account. A
2 of 3 scheme has the best characteristics for creating new bitcoin
addresses and for secure storing and spending. One compromised signatory
does not compromise the funds. A single secret key can be lost and
the funds can still be recovered. If done correctly, off-site backups
are created during wallet setup. The way to recover funds is known
by more than one party. The balance of power with a multi-signature
wallet can be shifted by having one party control more keys than the
other parties. If one party controls multiple keys, there is a greater
risk of those keys not remaining as multiple factors. \linebreak{}
\par\end{flushleft}

\noindent \begin{flushleft}
When $m=0.5n$, it is referred to as a split account, and is an interesting
use case, as there would be 3 of 6 where one person holds 3 keys and
3 people hold 1 key. In this way one person could control their own
money, but the funds could still be recoverable even if the primary
key holder were to disappear with all of his key. As $n$ increases,
the level of trust in the secondary parties can decrease. A good use
case might be a family savings account that would just automatically
become an inheritance account if the primary account holder were to
die. \cite{Contributors2017}
\par\end{flushleft}

\subsection{Recap on the Schnorr signature scheme }
\noindent \begin{flushleft}
The Schnorr signature scheme uses:\cite{Schnorr1991}
\par\end{flushleft}
\begin{itemize}
\item \begin{flushleft}
A cyclic group $G$ of prime order $p$
\par\end{flushleft}
\item \begin{flushleft}
A generator $g$of $G$
\par\end{flushleft}
\item \begin{flushleft}
A hash function $\textrm{H}$
\par\end{flushleft}
\item \begin{flushleft}
A private/public key pair is a pair $(x,X)$$\epsilon\{0,...,p-1\}$$\mathsf{x}$$G$
where $X=g^{x}$
\par\end{flushleft}
\item \begin{flushleft}
To sign a message $m$, the signer draws a random integer $r$ in
$Z_{p},$ computes $R=g^{r}$, $c=\textrm{H}(X,R,m)$, and $s=r+cx$
\par\end{flushleft}
\item \begin{flushleft}
The signature is the pair $(R,s)$ , and its validity can be checked
by verifying whether $g^{s}=RX^{c}$\linebreak{}
\par\end{flushleft}

\end{itemize}
\noindent \begin{flushleft}
The above described is referred to as the so-called ``key-prefixed''
variant of the scheme, which sees the public key hashed together with
$R$ and $m$\cite{Bernstein2012}. This variant was thought to have
a better multi-user security bound than the classic variant \cite{Bernstein2015},
however in \cite{Kiltz2016} the key-prefixing was seen as unnecessary
to enable good multi-user security for Schnorr signatures. \linebreak{}
\par\end{flushleft}

\noindent \begin{flushleft}
For the development of the new Schnorr-based multi-signature scheme
\cite{Maxwell2018}, key-prefixing seemed a requirement for the security
proof to go through, despite not knowing the form of an attack. The
rationale also follows the process in reality, as messages signed
in Bitcoin always indirectly commits to the public key. 
\par\end{flushleft}

\subsubsection{Rogue attacks }
\noindent \begin{flushleft}
Rogue attacks are a significant concern when implementing multi-signature
schemes. Here a subset of corrupted singers, manipulate the public
keys computed as functions of the public keys of honest users, allowing
them to easily produce forgeries for the set of public keys (despite
them not knowing the associated secret keys). \linebreak{}
\par\end{flushleft}

\noindent \begin{flushleft}
Initial proposals from \cite{Li1994}, \cite{Harn1994}, \cite{Horster1995},
\cite{Ohta1991}, \cite{Langford1996}, \cite{Michels1996} and \cite{Ohta1999}
were thus undone before a formal model was put forward along with
a provably secure scheme from Micali \emph{et al}. \cite{Micali2001}.
Unfortunately, despite being provably secure their scheme is costly
and an impractical interactive key generation protocol. \cite{Maxwell2018}\linebreak{}
\par\end{flushleft}

\noindent \begin{flushleft}
A means of generically preventing rogue-key attacks is to make it
mandatory for users to prove knowledge (or possession) of the secret
key during public key registration with a certification authority
\cite{Ristenpart2007}. Certification authority is a setting known
as the KOSK assumption. The pairing-based multi-signature schemes
by Boldyreva \cite{Boldyreva2003} and Lu \emph{et al}. \cite{Lu2006}
rely on the KOSK assumption in order to maintain security. However
according to\cite{Bellare2006} and\cite{Ristenpart2007} the cost
of complexity and expense of the scheme and the unrealistic and burdensome
assumptions on the public-key infrastructure (PKI) have made this
solution problematic. \linebreak{}
\par\end{flushleft}

\noindent \begin{flushleft}
As it stands, the Bellare and Neven \cite{Bellare2006} provides one
of the most practical multi-signature schemes, based on the Schnorr
signature scheme, which is provably secure that does not contain any
assumption on the key setup. Since the only requirement of this scheme
is that each potential signer has a public key, this setting is referred
to as the \emph{plain-key model. }
\par\end{flushleft}

\subsection{Design of a Schnorr multi-signature scheme}
\noindent \begin{flushleft}
The naive way to design a Schnorr multi-signature scheme would be
as follows:
\par\end{flushleft}
\begin{itemize}
\item \begin{flushleft}
A group of $n$ signers want to cosign a message $m$
\par\end{flushleft}
\item \begin{flushleft}
Let $L=\{X_{1}=g^{x_{1}},...,X_{n}=g^{x_{n}}\}$ be the multi-set
of all public key\footnote{No constraints are imposed on the key setup, the adversary thus can
choose corrupted public keys at random, hence the same public key
can appear more than once in $L$} 
\par\end{flushleft}
\item \begin{flushleft}
Each cosigner randomly generates and communicates to others a share
$R_{i=}$$g^{r_{i}}$
\par\end{flushleft}
\item \begin{flushleft}
Each of the cosigners then computes $R=$$\Pi_{_{i=1}}^{n}$$R_{i}$,
$c=\textrm{H}(\tilde{X},R,m)$ 
\par\end{flushleft}
\item \begin{flushleft}
Where $\tilde{X}=\Pi_{i=1}^{n}X_{i}$ is the product of individual
public keys, and a partial signature $s_{i}=r_{i}+cx_{i}$
\par\end{flushleft}
\item \begin{flushleft}
Partial signatures are then combined into a single signature $(R,s)$
where $s=\Sigma_{i=1}^{n}si$ mod $p$
\par\end{flushleft}
\item \begin{flushleft}
The validity of a signature $(R,s)$ on message $m$ for public keys
$\{X_{1},...X_{n}\}$ is equivalent to $g^{s}=R\tilde{X}^{c}$ where
$\tilde{X}=\Pi_{i=1}^{n}X_{i}$ and $c=\textrm{H}(\tilde{X},R,m)$\linebreak{}
\par\end{flushleft}

\end{itemize}
\noindent \begin{flushleft}
Note that this is exactly the verification equation for a traditional
key-prefixed Schnorr signature with respect to public key $\tilde{X}$,
a property termed \emph{key aggregation }
\par\end{flushleft}

\noindent \begin{flushleft}
However, as mentioned above, \cite{Horster1995}, \cite{Langford1996},
\cite{Michels1996} and\cite{Micali2001} these protocols are vulnerable
to a rogue-key attack where a corrupted signer sets its public key
to $X_{1}=g^{x_{1}}(\Pi_{i=2}^{n}X_{i})^{-1}$, allowing the signer
to produce signatures for public keys $\{X_{1},...X_{n}\}$ by themself.
\par\end{flushleft}

\subsubsection{Micali-Ohta-Reyzin Multi-signature scheme }
\noindent \begin{flushleft}
The Micali-Ohta-Reyzin multi-signature scheme \cite{Micali2001} solves
the rogue-key attack using a sophisticated interactive key generation
protocol. 
\par\end{flushleft}

\subsubsection{The Bagherzandi \emph{et al. }scheme}
\noindent \begin{flushleft}
Bagherzandi \emph{et al.} \cite{Bagherzandi2008} reduced the number
of rounds from three to two using an homomorphic commitment scheme.
Unfortunately, this increases the signature size and the computational
cost of signing and verification. 
\par\end{flushleft}

\subsubsection{The Ma \emph{et al.} scheme }
\noindent \begin{flushleft}
Ma \emph{et al.} \cite{Ma2010} proposed a signature scheme that involved
the ``double hashing'' technique, which sees the reduction of the
signature size compared to Bagherzandi \emph{et al.} \cite{Bagherzandi2008}
while using only two rounds. \linebreak{}
\par\end{flushleft}

\noindent \begin{flushleft}
However, neither of these two variants allow for key aggregation.
\linebreak{}
\par\end{flushleft}

\noindent \begin{flushleft}
Multi-signature schemes supporting key aggregation are easier to come
by in the KOSK model. In particular, Syta \emph{et al.} \cite{Syta2016}
proposed\emph{ }the CoSi scheme which can be seen as the naive Schnorr
multi-signature scheme described earlier where the co-signers are
organized in a tree structure for fast signature generation. 
\par\end{flushleft}

\subsubsection{Bellare and Neven signature scheme}
\noindent \begin{flushleft}
Bellare-Neven (BN) \cite{Bellare2006} proceeded differently in order
to avoid any key setup. Their main idea is to have each cosigner use
a distinct ``challenge'' when computing their partial signature
$s_{i}=r_{i}+c_{i}x_{i},$ defined as $c_{i}=\textrm{H}(<L>X_{i},R,m)$,
where $R=\prod_{i=1}^{n}R_{i}$ and $<L>$ is a unique encoding of
the multiset of public keys $L=\{X_{1}...X_{n}\}$. It is a more widely
known \emph{plain public-key }multi-signature scheme, that does not
support key aggregation. It is possible to use BN multi-signatures
where the individual keys are MuSig aggregates. BN multi-signature
scheme is secure without such assumptions. Below are details:
\par\end{flushleft}
\begin{itemize}
\item \begin{flushleft}
Call $L=\textrm{H}(X_{1,}X_{2...})$ 
\par\end{flushleft}
\item \begin{flushleft}
Each signer chooses a random nonce $r_{i}$ and shares $R$$_{i}=r_{i}G$
with the other signers
\par\end{flushleft}
\item \begin{flushleft}
Call R the sum of the $R_{i}$ points 
\par\end{flushleft}
\item \begin{flushleft}
Each signer computes $s_{i}=r_{i}+\textrm{H}(L,X_{i,}R,m)x_{i}$
\par\end{flushleft}
\item \begin{flushleft}
The final signature is $(R,s)$ where $s$is the sum of the $s_{i}$
values 
\par\end{flushleft}
\item \begin{flushleft}
Verification requires $sG=R+\textrm{H}(L,X_{1,}R,m)X_{2}+...$
\par\end{flushleft}

\end{itemize}
\noindent \begin{flushleft}
Technically, BN has a pre-commit round, where the signers initially
reveal $\textrm{H}(R_{i})$ to each other, prior to revealing the
$R_{i}$ points themselves. This step is a requirement in order to
prove security under the DL assumption, but it can be dismissed if
instead the OMDL assumption is accepted. \linebreak{}
\par\end{flushleft}

\noindent \begin{flushleft}
Furthermore, when an IAS is desired (where each signer has their own
message), $L=\textrm{H}((X_{1},m_{1}),(X_{2},m_{2}),...)$ and $s_{i}=r_{i}+\textrm{H}(L,R,i)x_{i}$
is used for signing (and analogous for verification). \linebreak{}
\par\end{flushleft}

\noindent \begin{flushleft}
The resulting signature does not satisfy the normal Schnorr equation
anymore, nor any other equation that can be written as a function
of a combination of the public keys; the key aggregation property
is lost in order to gain security in the \emph{plain public-key model}.
\linebreak{}
\par\end{flushleft}

\noindent \begin{flushleft}
Bellare and Neven showed that this yields a multi-signature scheme
provably secure in the \emph{plain public-key }model under the Discrete
Logarithm assumptions, modeling $\textrm{H}$ and $\textrm{H}'$ as
random oracles. However, this scheme does not allow key aggregation
anymore since the entire list of public keys is required for verification. 
\par\end{flushleft}

\section{The formation of MuSig }
\noindent \begin{flushleft}
This is where MuSig comes in. It recovers the \emph{key aggregation
property without losing security:}
\par\end{flushleft}
\begin{itemize}
\item \begin{flushleft}
Call $L=\textrm{H}(X_{1},X_{2}...)$
\par\end{flushleft}
\item \begin{flushleft}
Call $X$ the sum of all $\textrm{H}(L,X_{i})X_{i}$
\par\end{flushleft}
\item \begin{flushleft}
Each signer chooses a random nonce $r_{i},$ and shares $R$$_{i}=r_{i}G$
with the other signers 
\par\end{flushleft}
\item \begin{flushleft}
Call $R$ the sum of the $R$$_{i}$ points 
\par\end{flushleft}
\item \begin{flushleft}
Each signer computes $s_{i}=r_{i}+\textrm{H}(X,R,m)\textrm{H}(L,X_{i})x_{i}$
\par\end{flushleft}
\item \begin{flushleft}
The final signature is $(R,s)$ where $s$ is the sum of the $s_{i}$
values 
\par\end{flushleft}
\item \begin{flushleft}
Verification again satisfies $sG=R+\textrm{H}(X,R,m)X$
\par\end{flushleft}

\end{itemize}
\noindent \begin{flushleft}
So what was needed was to define $X$ not as a simple sum of the individual
public keys $X_{i}$, but as a sum of multiples of those keys, where
the multiplication factor depends on a hash of all participating keys.
\cite{Wuille2018}\linebreak{}
\par\end{flushleft}

\noindent \begin{flushleft}
The new proposed Schnorr-based multi-signature scheme can be seen
as a variant of the BN scheme, allowing key aggregation in the \emph{plain
public-key model. }This scheme consists of three rounds, the first
two being exactly the same as in BN. Challenges $c_{i}$ are changed
from $c_{i}=\textrm{H}(<L>X_{i},R,m)$ to $c_{i}=\textrm{H}_{agg}(<L>X_{i}).\textrm{H}_{sig}(\tilde{X,}R,m)${]}
, where $\tilde{X}$ is the so-called aggregated public key corresponding
to the multi-set of public keys $L=\{X_{1},...X_{n}\},$ defined as
$\tilde{X}=\stackrel[i=1]{n}{\Pi}X_{i}^{a_{i}c}=R\tilde{X}^{c}$ where
$c=\textrm{H}_{sig}(\tilde{X},R,m).$\linebreak{}
\par\end{flushleft}

\noindent \begin{flushleft}
Basically , the key aggregation property has been recovered and can
now be enjoyed by the naive scheme, which respect to a more complex
aggregation key $\tilde{X}=\stackrel[i=1]{n}{\Pi}X_{i}^{a_{i}c}$. 
\par\end{flushleft}

\noindent \begin{flushleft}
$c=H_{sig}(<L>,R,m)$ yields a secure scheme, however does not allow
key aggregation since verification is impossible without knowing all
the individual singer keys. 
\par\end{flushleft}

\subsection{Interactive Aggregate Signatures}
\noindent \begin{flushleft}
In some situations, it may be useful to allow each participant to
sign a different message rather than a single common one. An IAS is
one where each signer has its own message \emph{$m_{i}$ }to sign,
and the joint signature proves that the $i$-th signer has signed
$m_{i}$ . These schemes are considered to be more general than multi-signature
schemes, however they are not as flexible as non-interactive aggregate
signatures (\cite{Boneh2003}, \cite{Bellare2007}) and sequential
aggregate signatures \cite{Lysyanskaya2004}. \linebreak{}
\par\end{flushleft}

\noindent \begin{flushleft}
According to Bellare \emph{et al.} \cite{Bellare2006}, a generic
way to turn any multi-signature scheme into an IAS scheme by the signer
running the multi-signature protocol using as message the tuple of
all public keys/message pairs involved in the IAS protocol. \linebreak{}
\par\end{flushleft}

\noindent \begin{flushleft}
For BN's scheme and Schnorr multi-signatures, this does not increase
the number of communication rounds as messages can be sent together
with shares $R_{i}$. 
\par\end{flushleft}

\subsubsection{Applications of IAS}
\noindent \begin{flushleft}
With regards to digital currency schemes, where all participants have
the ability to validate transactions, these transactions consist of
outputs (which have a verification key and amount) and inputs (which
are references to outputs of earlier transactions). Each input contains
a signature of a modified version of the transaction to be validated
with its referenced output's key. Some outputs may require multiple
signatures to be spent. Transactions spending such an output are referred
to as \emph{m}-of-\emph{n }multi-signature transactions \cite{Andersen2011},
and the current implementation corresponds to the trivial way of building
a multi-signature scheme by concatenating individual signatures. Additionally,
a threshold policy can be enforced where only $m$ valid signatures
out of the $n$ possible ones are needed to redeem the transaction
(again this is the most straightforward way to turn a multi-signature
scheme into some kind of basic threshold signature scheme).\linebreak{}
\par\end{flushleft}

\noindent \begin{flushleft}
While several multi-signature schemes could offer an improvement over
the currently available method, two properties increase the possible
impact:
\par\end{flushleft}
\begin{itemize}
\item \begin{flushleft}
The availability of key aggregation removes the need for verifiers
to see all the involved key, improving bandwidth, privacy, and validation
cost
\par\end{flushleft}
\item \begin{flushleft}
Security under the \emph{plain public-key model }enables multi-signatures
across multiple inputs of a transaction, where the choice of signers
cannot be committed to in advance. This greatly increases the number
of situations in which mulit-signatures are beneficial.
\par\end{flushleft}
\end{itemize}

\subsubsection{Native multi-signature support }
\noindent \begin{flushleft}
An improvement is to replace the need for implementing \emph{$n-of-n$
}multi-signatures with a constant-size multi-signature primitive like
Bellare-Neven. While this is on itself an improvement in terms of
size, it still needs to contain all of the signers' public keys. Key
aggregation improves upon this further, as a single-key predicate\footnote{Predicate encryption is an encryption paradigm which gives a master
secret key owner fine-grained control over access to encrypted data.\cite{Shen2009} } can be used instead which is both smaller and has lower computational
cost for verification. It also improves privacy, as the participant
keys and their count remain private to the signers. \linebreak{}
\par\end{flushleft}

\noindent \begin{flushleft}
When generalizing to the $m-of-n$ scenario, several options exist.
One is to forego key aggregation, and still include all potential
signer keys in the predicates while still only producing a single
signature for the chosen combination of keys. Alternatively, a Merkle
tree \cite{Merkle1987} where the leaves are permitted combinations
of public keys (in aggregated form), can be employed. The predicate
in this case would take as input an aggregated public key, a signature
and a proof. Its validity would depend on the signature being valid
with the provided key, and the proof establishing that the key is
in fact one of the leaves of the Merkle tree, identified by its root
hash. This approach is very generic, as it works for any subset of
combinations of keys, and as a result has good privacy as the exact
policy is not visible from the proof. \linebreak{}
\par\end{flushleft}

\noindent \begin{flushleft}
Some key aggregation schemes that do not protect against rogue-key
attacks can be used instead in the above cases, under the assumption
that the sender is given a proof of knowledge/possession for the receivers'
private keys. However, these schemes are difficult to prove secure
except by using very large proofs of knowledge. As those proofs of
knowledge/possession do not need to be seen by verifiers, they are
effectively certified by the senders's validation. However, passing
them around to senders is inconvenient, and easy to get wrong. Using
a scheme that is secure in the \emph{plain public-key model }categorically
avoids these concerns. \linebreak{}
\par\end{flushleft}

\noindent \begin{flushleft}
Another alternative is to use an algorithm whose key generation requires
a trusted setup, for example in the KOSK model. While many of these
schemes have been proven secure \cite{Boldyreva2003}\cite{Lu2006},
they rely on mechanisms that are usually not implemented by certification
authorities \cite{Bellare2006} \cite{Ristenpart2007}. 
\par\end{flushleft}

\subsubsection{Cross-input multi-signatures }
\noindent \begin{flushleft}
The previous sections explained how the numbers of signatures per
input can generally by reduced to one, but one can go further and
replace it with a single signature per transaction. Doing so requires
a fundamental change in validation semantics, as the validity of separate
inputs is no longer independent. As a result, the outputs can no longer
be modeled as predicates, where the secret key owner is given access
to encrypted data. Instead, they are modeled as functions that return
a boolean (data type with only two possible values) plus a set of
zero or more public keys. \linebreak{}
\par\end{flushleft}

\noindent \begin{flushleft}
Overall validity requires all returned booleans to be True and a multi-signature
of the transaction with $L$ the union of all returned keys. \linebreak{}
\par\end{flushleft}

\noindent \begin{flushleft}
With regards to Bitcoin, this can be implemented by providing an alternative
to the signature checking opcode OP\_CHECKSIG and related opcodes
in the Script language. Instead of returning the result of an actual
ECDSA verification, they always return True, but additionally add
the public key with which the verification would have taken place
to a transaction-wide multi-set of keys. Finally, after all inputs
are verified, a multi-signature present in the transaction is verified
against that multi-set. In case the transaction spends inputs from
multiple owners, they will need to collaborate to produce the multi-signature,
or choose to only use the original opcodes. Adding these new opcodes
is possible in a backward-compatible way. \cite{Maxwell2018}
\par\end{flushleft}

\subsubsection{Protection against rogue-key Attacks}
\noindent \begin{flushleft}
In Bitcoin, when taking cross-input signatures into account, there
is no published commitment to the set of signers, as each transaction
input can independently spend an output that requires authorization
from distinct participants. This functionality was not restricted
as it would then interfere with fungibility improvements such as CoinJoin
\cite{Maxwell2013}. Due to the lack of certification, security against
rogue-key attacks is of great importance. \linebreak{}
\par\end{flushleft}

\noindent \begin{flushleft}
If it is assumed that transactions used a single multi-signature that
was vulnerable to rogue-attacks, an attacker could identify an arbitrary
number of outputs he wants to steal, with the public keys $X_{1},...,X_{n-t}$
and then use the rogue-key attack to determine $X_{n-t+1},...,X_{n}$
such that he can sign for the aggregated key $\tilde{X}.$ He would
then send a small amount of his own money to outputs with predicates
corresponding to the keys $X_{n-t+1},...,X_{n}$. Finally, he can
create a transaction that spends all of the victim coins together
with the ones he just created by forging a multi-signature for the
whole transaction. \linebreak{}
\par\end{flushleft}

\noindent \begin{flushleft}
It can be seen that in the case of mulit-signatures across inputs,
theft can occur through the ability to forge a signature over a set
of keys that includes at least one key which is not controlled by
the attacker. According to the \emph{plain public-key model }this
is considered a win for the attacker. This is in contrast to the single-input
multi-signature case where theft is only possible by forging a signature
for the exact (aggregated) keys contained in an existing output. As
a result, it is no longer possible to rely on proofs of knowledge/possession
that are private to the signers. 
\par\end{flushleft}

\subsection{Revisions }
\noindent \begin{flushleft}
In a previous version of the paper by Maxwell \emph{et al.} \cite{Maxwell2018}\emph{
}published on 15 January 2018 they proposed a 2-round variant of MuSig,
where the initial commitment round is omitted claiming a security
proof under the One More Discrete Logarithm (OMDL) assumptions (\cite{Bellare2002},
\cite{Bellare2003}). Drijvers \emph{et al }\cite{Drijvers2018} then
discovered a flaw in the security proof and showed that through a
meta-reduction the initial multi-signature scheme cannot be proved
secure using an algebraic black box reduction under the DL or OMDL
assumption. \linebreak{}
\par\end{flushleft}

\noindent \begin{flushleft}
In more details, it was observed that in the 2-round variant of MuSig,
an adversary (controlling public keys $X_{2},...,X_{n})$ can impose
the value of $R=\Pi_{i=1}^{n}R_{i}$ used in signature protocols since
he can choose $R_{2},...,R_{n}$ after having received $R_{1}$ from
the honest signer (controlling public key $X_{1}=g^{x_{1}}$). This
prevents one to use the initial method of simulating the honest signer
in the Random Oracle model without knowing $x_{1}$ by randomly drawing
$s_{1}$ and $c$, computing 
\par\end{flushleft}

\noindent \begin{flushleft}
Despite this, there is no attack currently known against the 2-round
variant of MuSig and that it might be secure, although this is not
provable under standard assumptions from existing techniques. \cite{Maxwell2018}
\par\end{flushleft}

\section{Conclusions, Observations and Recommendations}
\begin{itemize}
\item \begin{flushleft}
MuSig leads to both native and private multi-signature transactions
with both signature aggregation. 
\par\end{flushleft}
\item \begin{flushleft}
Signature data for multi-signatures can be large and cumbersome. MuSig
will allow users to create more complex transactions without burdening
the network and revealing compromising information. 
\par\end{flushleft}
\item \begin{flushleft}
However, the case of interactive signature aggregation where each
signer signs their own message must still be proven by a complete
security analysis. 
\par\end{flushleft}
\end{itemize}

\section{Contributors }

\noindent Kevoulee Sardar, Hansie Odendaal 

\paragraph*{\newpage\bibliographystyle{ieeetr}
\bibliography{MuSig/musig}
}
\end{document}
